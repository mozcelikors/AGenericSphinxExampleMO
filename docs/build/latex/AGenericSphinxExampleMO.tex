%% Generated by Sphinx.
\def\sphinxdocclass{report}
\documentclass[letterpaper,10pt,english]{sphinxmanual}
\ifdefined\pdfpxdimen
   \let\sphinxpxdimen\pdfpxdimen\else\newdimen\sphinxpxdimen
\fi \sphinxpxdimen=.75bp\relax

\usepackage[utf8]{inputenc}
\ifdefined\DeclareUnicodeCharacter
 \ifdefined\DeclareUnicodeCharacterAsOptional
  \DeclareUnicodeCharacter{"00A0}{\nobreakspace}
  \DeclareUnicodeCharacter{"2500}{\sphinxunichar{2500}}
  \DeclareUnicodeCharacter{"2502}{\sphinxunichar{2502}}
  \DeclareUnicodeCharacter{"2514}{\sphinxunichar{2514}}
  \DeclareUnicodeCharacter{"251C}{\sphinxunichar{251C}}
  \DeclareUnicodeCharacter{"2572}{\textbackslash}
 \else
  \DeclareUnicodeCharacter{00A0}{\nobreakspace}
  \DeclareUnicodeCharacter{2500}{\sphinxunichar{2500}}
  \DeclareUnicodeCharacter{2502}{\sphinxunichar{2502}}
  \DeclareUnicodeCharacter{2514}{\sphinxunichar{2514}}
  \DeclareUnicodeCharacter{251C}{\sphinxunichar{251C}}
  \DeclareUnicodeCharacter{2572}{\textbackslash}
 \fi
\fi
\usepackage{cmap}
\usepackage[T1]{fontenc}
\usepackage{amsmath,amssymb,amstext}
\usepackage{babel}
\usepackage{times}
\usepackage[Bjarne]{fncychap}
\usepackage[dontkeepoldnames]{sphinx}

\usepackage{geometry}

% Include hyperref last.
\usepackage{hyperref}
% Fix anchor placement for figures with captions.
\usepackage{hypcap}% it must be loaded after hyperref.
% Set up styles of URL: it should be placed after hyperref.
\urlstyle{same}
\addto\captionsenglish{\renewcommand{\contentsname}{Contents:}}

\addto\captionsenglish{\renewcommand{\figurename}{Fig.}}
\addto\captionsenglish{\renewcommand{\tablename}{Table}}
\addto\captionsenglish{\renewcommand{\literalblockname}{Listing}}

\addto\captionsenglish{\renewcommand{\literalblockcontinuedname}{continued from previous page}}
\addto\captionsenglish{\renewcommand{\literalblockcontinuesname}{continues on next page}}

\addto\extrasenglish{\def\pageautorefname{page}}

\setcounter{tocdepth}{1}



\title{AGenericSphinxExampleMO\ Documentation}
\date{Jul 24, 2017}
\release{1.01}
\author{Mustafa Özçelikörs}
\newcommand{\sphinxlogo}{\vbox{}}
\renewcommand{\releasename}{Release}
\makeindex

\begin{document}

\maketitle
\sphinxtableofcontents
\phantomsection\label{\detokenize{index::doc}}



\chapter{Introduction}
\label{\detokenize{intro:introduction}}\label{\detokenize{intro:welcome-to-a-generic-sphinx-example-documentation}}\label{\detokenize{intro::doc}}

\section{Title}
\label{\detokenize{intro:title}}
Sphinx is an amazing thing! Here is the code.. This is an \sphinxhref{http://thewebblog.net}{inline link}. See the figure from the reference here {\hyperref[\detokenize{intro:reference-label}]{\sphinxcrossref{\DUrole{std,std-ref}{Figure text}}}}. See this table too: {\hyperref[\detokenize{intro:reference-label2}]{\sphinxcrossref{\DUrole{std,std-ref}{Table Title}}}}. You can see that \sphinxstylestrong{this is bold text} while \sphinxstyleemphasis{this is italic} and this is \sphinxcode{verbatim, ergo the highlighted text}. This is from the citation \phantomsection\label{\detokenize{intro:id1}}{\hyperref[\detokenize{intro:citation}]{\sphinxcrossref{{[}CITATION{]}}}}.

\begin{sphinxVerbatim}[commandchars=\\\{\},numbers=left,firstnumber=1,stepnumber=1]
\PYG{c+cp}{\PYGZsh{}}\PYG{c+cp}{include} \PYG{c+cpf}{\PYGZlt{}stdint.h\PYGZgt{}}
\PYG{n}{myFunction} \PYG{p}{(}\PYG{k+kt}{void}\PYG{p}{)}
\PYG{p}{\PYGZob{}}
    \PYG{k+kt}{int} \PYG{n}{x}\PYG{p}{;}
\PYG{p}{\PYGZcb{}}
\end{sphinxVerbatim}

\begin{figure}[htbp]
\centering
\capstart

\noindent\sphinxincludegraphics[width=200\sphinxpxdimen,height=100\sphinxpxdimen]{{stars}.jpg}
\caption{Figure text}\label{\detokenize{intro:reference-label}}\label{\detokenize{intro:id2}}\end{figure}

Now, here is a table:


\begin{savenotes}\sphinxattablestart
\centering
\sphinxcapstartof{table}
\sphinxcaption{Table Title}\label{\detokenize{intro:reference-label2}}\label{\detokenize{intro:id3}}
\sphinxaftercaption
\begin{tabular}[t]{|\X{20}{50}|\X{20}{50}|\X{10}{50}|}
\hline
\sphinxstylethead{\sphinxstyletheadfamily 
name
\unskip}\relax &\sphinxstylethead{\sphinxstyletheadfamily 
firstname
\unskip}\relax &\sphinxstylethead{\sphinxstyletheadfamily 
age
\unskip}\relax \\
\hline
Smith
&
John
&
40
\\
\hline
Smith
&
John Jr.
&
20
\\
\hline
\end{tabular}
\par
\sphinxattableend\end{savenotes}

Aaaand here is a note box:

\begin{sphinxadmonition}{note}{Note:}
This is a \sphinxstylestrong{note} box.
\end{sphinxadmonition}

And this is a warning box:

\begin{sphinxadmonition}{warning}{Warning:}
This is a warning box..
\end{sphinxadmonition}

And this is a seealso box:


\sphinxstrong{See also:}


This is a seealso box..



\begin{sphinxShadowBox}
\sphinxstyletopictitle{Your custom Topic Box}

Lorem ipsum etc etc.
\end{sphinxShadowBox}


\subsection{Subtitle}
\label{\detokenize{intro:subtitle}}

\subsubsection{subsubtitle}
\label{\detokenize{intro:subsubtitle}}
Define citations at the bottom of the page


\chapter{Second Chapter}
\label{\detokenize{chapter1:second-chapter}}\label{\detokenize{chapter1::doc}}

\chapter{Third Chapter}
\label{\detokenize{chapter2::doc}}\label{\detokenize{chapter2:third-chapter}}

\chapter{Indices and tables}
\label{\detokenize{index:indices-and-tables}}\begin{itemize}
\item {} 
\DUrole{xref,std,std-ref}{genindex}

\item {} 
\DUrole{xref,std,std-ref}{modindex}

\item {} 
\DUrole{xref,std,std-ref}{search}

\end{itemize}

\begin{sphinxthebibliography}{CITATION}
\bibitem[CITATION]{\detokenize{CITATION}}{\phantomsection\label{\detokenize{intro:citation}} 
A citations
(etc etc)
}
\end{sphinxthebibliography}



\renewcommand{\indexname}{Index}
\printindex
\end{document}